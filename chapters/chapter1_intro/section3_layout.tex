\section{Διάρθρωση της Αναφοράς}
\label{section:layout}

Η διάρθρωση της παρούσας διπλωματικής εργασίας είναι η εξής:

\begin{itemize}
  \item{\textbf{Κεφάλαιο \ref{chapter:sota}:}
      Γίνεται ανασόπιση της ερευνητικής περιοχής που αφορά την αναγνώριση και εντοπισμό αντικειμένων με χρήση τεχνικών μηχανικής μάθησης και πιο συγκεκριμένα με χρήση μοντέλων CNN.
    }
  \item{\textbf{Κεφάλαιο \ref{chapter:theory}:} Περιγράφονται τα βασικά θεωρητικά στοιχεία
      στα οποία βασίστηκαν οι υλοποιήσεις. Γίνεται εισαγωγή στην επιστήμη
      της μηχανικής μάθησης και καταλήγει στην περιγραφή της λειτουργίας
      και χρήσης των CNN.
    }
  \item{\textbf{Κεφάλαιο \ref{chapter:tools}:} Παρουσιάζονται οι
      διάφορες τεχνικές και τα εργαλεία που χρησιμοποιήθηκαν στις
      υλοποιήσεις.
    }
  \item{\textbf{Κεφάλαιο \ref{chapter:implementations}:} Πλήρης περιγραφή των υλοποιήσεων
      διαφόρων μοντέλων CNN (AlexNet, VGG16, Tiny-YOLO).
    }
  \item{\textbf{Κεφάλαιο \ref{chapter:experiments}:} Παρουσιάζεται αναλυτικά η μεθοδολογία των
      πειραμάτων και τα αποτελέσματα.
    }
  \item{\textbf{Κεφάλαιο \ref{chapter:conclusions}:} Παρουσιάζονται τα τελικά συμπεράσματα.
    }
  \item{\textbf{Κεφάλαιο \ref{chapter:future_work}:} Αναφέρονται τα
      προβλήματα που προέκυψαν και προτείνονται θέματα για μελλοντική
      μελέτη, αλλαγές και επεκτάσεις.
    }
\end{itemize}

