\section{Περιγραφή του Προβλήματος}
\label{section:problem_description}

Παρόλο που τα CΝΝ είναι ικανά να δώσουν λύσεις με μεγάλη ακρίβεια, απαιτούν
μεγάλη επεξεργαστική ισχύ τόσο για την εκπαίδευσή τους όσο και για την
εκτέλεση ενός πειράματος, ιδιαίτερα όταν το πρόβλημα για το οποίο έχουν
σχεδιαστεί να δώσουν λύση είναι περίπλοκο.
Η ανάγκη μεγάλης επεξεργαστικής ισχύος οφείλεται στο μεγάλο αριθμό επιπέδων
από τα οποία αποτελούνται τα μοντέλα CNN.
Για παράδειγμα, ένα από τα πρώτα μοντέλα CNN το οποίο σχεδιάστηκε για την
αναγνώριση αντικειμένων σε εικόνες, αποτελείται από 16 επίπεδα (AlexNet)
και έχει εξήντα εκατομμύρια (60,000,000) παραμέτρους και
εξακόσιες πενήντα χιλιάδες (650,000) νευρώνες από τους οποίους οι περισσότεροι
εκτελούν πράξεις συνέλιξης. Ο Alex Krizhevsky απέδειξε το 2012 ότι η χρήση
σύγχρονων μονάδων GPU για την εκτέλεση πράξεων συνέλιξης, έχει ως αποτέλεσμα την
εκπαίδευση μοντέλων CNN σε χρόνους έως και δύο τάξεις μεγέθους πιο κάτω σε σύγκριση
με μία ισχυρή επεξεργαστή μονάδα CPU \cite{NIPS2012_4824}.
Ο χρόνος εκτέλεσης ενός πειράματος του δικτύου AlexNet
έχει μετρηθεί στα 7.39 δευτερόλεπτα σε οκταπύρηνο επεξεργαστή Haskwell @2.9Ghz
και στα 0.71 δευτερόλεπτα σε μονάδα GPU, Nvidia K520 \cite{abuzaidoptimizing}.

Είναι ιδιαίτερα σημαντικό ένα ρομπότ να μπορεί να αντιλαμβάνεται το περιβάλλον
του.
Αυτό περιλαμβάνει την αναγνώριση ανθρώπων, ζώων και αντικειμένων γενικότερα. Ωστόσο,
θέλουμε τα ρομπότ να είναι όσο πιο “ελκυστικά” γίνεται και συνήθως μικρότερα σε μέγεθος από τον άνθρωπο (uncunny valley in robotics) \cite{mori2012uncanny},
ανάλογα με το task που επιθυμούμε να εκτελέσουν.
Αυτό, έχει ως αποτέλεσμα να μην μπορούμε να τοποθετήσουμε ογκώδη, άρα με μεγάλη
επεξεργαστική ισχύ, υπολογιστικά συστήματα στο σώμα των ρομποτικών συστημάτων σε όλες
τις περιπτώσεις.

Παρόλο που σήμερα έχουν σχεδιαστεί μοντέλα CNN τα οποία έχουν την δυνατότητα να
αναγνωρίσουν και να εντοπίσουν αντικείμενα ανάμεσα σε χιλιάδες
κλάσεις, ο χρόνος εκτέλεσής τους
είναι αρκετά μεγάλος (της τάξης των μερικών δευτερολέπτων σε σύγχρονες υπολογιστικές μονάδες CPU).
Αυτό κάνει την χρήση των CNN σε εφαρμογές πραγματικού χρόνου, όπως για παράδειγμα
στην ρομποτική, ακατάλληλη.
Ωστόσο, η επιστημονική κοινότητα σήμερα προσπαθεί να δώσει λύσεις στο συγκεκριμένο
πρόβλημα εστιάζοντας το ενδιαφέρον στην εξέλιξη των ενσωματωμένων συστημάτων
και την σχεδίαση γρήγορου λογισμικού για υλοποιήσεις μοντέλων CNN τα οποία
εκμεταλλεύονται κυρίως την υπολογιστική ισχύ των μονάδων GPU, αλλά και άλλων
πολυπύρηνων επεξεργαστικών μονάδων.
