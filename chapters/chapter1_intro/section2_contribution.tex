\section{Σκοπός - Συνεισφορά της Διπλωματικής Εργασίας}
\label{section:contribution}

Η παρούσα διπλωματική εργασία μελετά την χρήση νευρωνικών δικτύων συνέλιξης (CNN)
σε εφαρμογές ταυτόχρονης αναγνώρισης και εντοπισμού αντικειμένων
(object recognition and localization - object detection) σε εικόνες.

Στοχεύει στην υλοποίηση μοντέλων CNN και στην εφαρμογή τους σε προβλήματα πραγματικού (σχεδόν)
χρόνου, όπου μία από της απαιτήσεις είναι ο γρήγορος χρόνος εκτέλεσης.

Εξετάζεται η ανάπτυξή μοντέλων CNN (για εφαρμογές αναγνώρισης και εντοπισμού αντικειμένων σε εικόνες)
σε ένα σταθερό υπολογιστικό σύστημα (PC)
καθώς και η χρήση υβριδικών ενσωματωμένων συστημάτων, τα οποία
φέρουν μονάδες CPU και GPU (όπως για παράδειγμα το Jetson TK1).

Επίσης παρουσιάζει μία σειρά από διαδικασίες βελτιστοποίησης για το
ενσωματωμένο σύστημα Jetson TK1, με στόχο την γρήγορη εκτέλεση
της διαδικασίας προς-τα-εμπρός περάσματος (forward propagation) στο εκάστοτε νευρωνικό δίκτυο,
καθώς και την μείωση της κατανάλωσης ισχύος κατά την διαδικασία εκτέλεσης.

% συνεισφορά 1: survey σχετικό με deep learning για object detection & εφαμρογή τους σε PC & K1
% συνεισφορά 2: Optimization ενός DNN για τον K1
