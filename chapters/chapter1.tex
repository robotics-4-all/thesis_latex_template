\chapter{Εισαγωγή}
\label{chapter:intro}

Ο όρος \emph{Τεχνητή Νοημοσύνη} αναφέρεται στην ικανότητα των υπολογιστικών
συστημάτων να μιμούνται στοιχεία της ανθρώπινης συμπεριφοράς.
Η επιθυμία των ανθρώπων να κατασκευάσουν “έξυπνες” μηχανές, καταγράφεται από
την εποχή της αρχαίας Ελλάδας. Μυθικές μορφές όπως οι Πυγμαλίωνας, Δαίδαλος και
Ήφαιστος μπορούν να θεωρηθούν ως εφευρέτες και δημιουργοί νοούντων
μηχανών ανάμεσα σε άλλες την Γαλάτεια, τον Τάλω και την Πανδώρα.
Ένας πιο ολοκληρωμένος ορισμός της Τεχνητής Νοημοσύνης είναι \cite{barr1989handbook}:
\begin{displayquote}
\emph{
  Ο κλάδος/τομέας της επιστήμης της πληροφορικής, που ασχολείται
  με την σχεδίαση και κατασκευή ευφυών συστημάτων, δηλαδή συστημάτων που
  διαθέτουν χαρακτηριστικά που σχετίζονται με την ανθρώπινη νοημοσύνη και συμπεριφορά.
}
\end{displayquote}

Με την εμφάνιση των πρώτων ηλεκτρονικών και (επανα)προγραμματιζόμενων υπολογιστικών συστημάτων,
οι άνθρωποι ξεκίνησαν να σκέφτονται τρόπους για να κατασκευάσουν “έξυπνες” μηχανές.
H ραγδαία εξέλιξη στον κλάδο της επιστήμης της πληροφορικής τις τελευταίες
δεκαετίες, έφερε και την εξέλιξη στην επιστήμη της Τεχνητής Νοημοσύνης.
Το 1997, η IBM κατασκεύασε ένα υπολογιστικό σύστημα που μπορούσε να
παίξει σκάκι (Deep Blue) \cite{campbell2002deep}, το οποίο κέρδισε τον παγκόσμιο %ref does not exist
τότε πρωταθλητή στο σκάκι, Garry Kasparov. Το σκάκι έχει εξήντα τέσσερις θέσεις
και τριάντα δύο πιόνια που μπορούν να κινούνται με συγκεκριμένο τρόπο. H μηχανή
Deep Blue, μπορούσε να εκτιμήσει και να αξιολογήσει διακόσια εκατομμύρια
πιθανές καταστάσεις της σκακιέρας. Ωστόσο, πρέπει να σημειωθεί ότι η επίλυση του
προβλήματος του σκακιού, παρ' ότι είναι ένα πρόβλημα το οποίο μπορεί να περιγραφεί
πλήρως μέσα από μια λίστα με κανόνες, δεν είναι δυνατό να λυθεί αναλυτικά εξαιτίας
του τεράστιου αριθμού των δυνατών κινήσεων.
%\begin{figure}[!ht]
  %\centering
  %\includegraphics[width=0.7\textwidth]{./images/chapter3/building_a_brain.jpg}
  %% \caption[Τεχνητή Νοημοσύνη]{Τεχνητή Νοημοσύνη}
  %\label{fig:AI_1}
%\end{figure}

Τα ρομποτικά συστήματα ήδη έχουν περάσει φάση δοκιμής, σε διάφορες χώρες ανά τον
κόσμο, όσον αφορά την ένταξή τους στην καθημερινότητα του ανθρώπου.
Η “δύναμη” της τεχνητής νοημοσύνης υπερβαίνει τα όρια της φαντασίας.

Κύριο χαρακτηριστικό ενός “ευφυούς” ρομποτικού συστήματος είναι να αντιλαμβάνεται
το περιβάλλον του.
Ένας ρομποτικός πράκτορας (robotic agent) για να πραγματοποιήσει μία πληθώρα
εφαρμογών, πρέπει να διαθέτει τόσο την
ικανότητα να αναγνωρίζει, μέσω εικόνων λήψης από κάμερα, διάφορα αντικείμενα,
όσο και την ικανότητα να εντοπίζει επακριβώς την θέση του
εκάστοτε αντικειμένου στον χώρο. Δεν θα είχε νόημα για ένα ρομπότ να έχει άποψη
για την παρουσία αντικειμένων γύρω του, αν δεν έχει και την ικανότητα να γνωρίζει
την θέση των αντικειμένων αυτών. % not exactly but...

Ένας σημαντικός παράγοντας της ραγδαίας εξέλιξης της επιστήμης της ρομποτικής, αλλά
και της τεχνητής νοημοσύνης γενικότερα, είναι η εμφάνιση και εξέλιξη του κλάδου
της \emph{Βαθιάς Μηχανικής Μάθησης} (Deep learning - DL).

Η χρήση τεχνικών βαθιάς μάθησης στην επίλυση προβλημάτων \emph{Μηχανικής Όρασης},
έχει φέρει την συγκεκριμένη τεχνολογία σε θέση να μπορεί να αντιμετωπίσει
περίπλοκα προβλήματα τα οποία μέχρι και πριν από λίγα χρόνια θεωρείτο ακατόρθωτο να λυθούν.
Ένα χαρακτηριστικό παράδειγμα είναι τα αυτόνομα αυτοκίνητα, τα οποία σήμερα είναι σε
φάση δοκιμής %\cite{gerla2014internet}.

Σήμερα, το γενικότερο πρόβλημα της ταυτόχρονης αναγνώρισης και εντοπισμού αντικειμένων σε εικόνες
λύνεται με την χρήση Νευρωνικών Δικτύων Συνέλιξης (Convolutional Neural Networks - CNNs).
Το γεγονός ότι τα CNΝs, σε συνδυασμό με μονάδες GPU, δίνουν την δυνατότητα
επίλυσης προβλημάτων αναγνώρισης και εντοπισμού αντικειμένων σε μικρό χρόνο,
τα καθιστά ικανά να χρησιμοποιηθούν σε εφαρμογές πραγματικού χρόνου.

\section{Περιγραφή του Προβλήματος}
\label{section:problem_description}

Παρόλο που τα CΝΝ είναι ικανά να δώσουν λύσεις με μεγάλη ακρίβεια, απαιτούν
μεγάλη επεξεργαστική ισχύ τόσο για την εκπαίδευσή τους όσο και για την
εκτέλεση ενός πειράματος, ιδιαίτερα όταν το πρόβλημα για το οποίο έχουν
σχεδιαστεί να δώσουν λύση είναι περίπλοκο.
Η ανάγκη μεγάλης επεξεργαστικής ισχύος οφείλεται στο μεγάλο αριθμό επιπέδων
από τα οποία αποτελούνται τα μοντέλα CNN.
Για παράδειγμα, ένα από τα πρώτα μοντέλα CNN το οποίο σχεδιάστηκε για την
αναγνώριση αντικειμένων σε εικόνες, αποτελείται από 16 επίπεδα (AlexNet)
και έχει εξήντα εκατομμύρια (60,000,000) παραμέτρους και
εξακόσιες πενήντα χιλιάδες (650,000) νευρώνες από τους οποίους οι περισσότεροι
εκτελούν πράξεις συνέλιξης. Ο Alex Krizhevsky απέδειξε το 2012 ότι η χρήση
σύγχρονων μονάδων GPU για την εκτέλεση πράξεων συνέλιξης, έχει ως αποτέλεσμα την
εκπαίδευση μοντέλων CNN σε χρόνους έως και δύο τάξεις μεγέθους πιο κάτω σε σύγκριση
με μία ισχυρή επεξεργαστή μονάδα CPU \cite{NIPS2012_4824}.
Ο χρόνος εκτέλεσης ενός πειράματος του δικτύου AlexNet
έχει μετρηθεί στα 7.39 δευτερόλεπτα σε οκταπύρηνο επεξεργαστή Haskwell @2.9Ghz
και στα 0.71 δευτερόλεπτα σε μονάδα GPU, Nvidia K520 \cite{abuzaidoptimizing}.

Είναι ιδιαίτερα σημαντικό ένα ρομπότ να μπορεί να αντιλαμβάνεται το περιβάλλον
του.
Αυτό περιλαμβάνει την αναγνώριση ανθρώπων, ζώων και αντικειμένων γενικότερα. Ωστόσο,
θέλουμε τα ρομπότ να είναι όσο πιο “ελκυστικά” γίνεται και συνήθως μικρότερα σε μέγεθος από τον άνθρωπο (uncunny valley in robotics) \cite{mori2012uncanny},
ανάλογα με το task που επιθυμούμε να εκτελέσουν.
Αυτό, έχει ως αποτέλεσμα να μην μπορούμε να τοποθετήσουμε ογκώδη, άρα με μεγάλη
επεξεργαστική ισχύ, υπολογιστικά συστήματα στο σώμα των ρομποτικών συστημάτων σε όλες
τις περιπτώσεις.

Παρόλο που σήμερα έχουν σχεδιαστεί μοντέλα CNN τα οποία έχουν την δυνατότητα να
αναγνωρίσουν και να εντοπίσουν αντικείμενα ανάμεσα σε χιλιάδες
κλάσεις, ο χρόνος εκτέλεσής τους
είναι αρκετά μεγάλος (της τάξης των μερικών δευτερολέπτων σε σύγχρονες υπολογιστικές μονάδες CPU).
Αυτό κάνει την χρήση των CNN σε εφαρμογές πραγματικού χρόνου, όπως για παράδειγμα
στην ρομποτική, ακατάλληλη.
Ωστόσο, η επιστημονική κοινότητα σήμερα προσπαθεί να δώσει λύσεις στο συγκεκριμένο
πρόβλημα εστιάζοντας το ενδιαφέρον στην εξέλιξη των ενσωματωμένων συστημάτων
και την σχεδίαση γρήγορου λογισμικού για υλοποιήσεις μοντέλων CNN τα οποία
εκμεταλλεύονται κυρίως την υπολογιστική ισχύ των μονάδων GPU, αλλά και άλλων
πολυπύρηνων επεξεργαστικών μονάδων.

\section{Σκοπός - Συνεισφορά της Διπλωματικής Εργασίας}
\label{section:contribution}

Η παρούσα διπλωματική εργασία μελετά την χρήση νευρωνικών δικτύων συνέλιξης (CNN)
σε εφαρμογές ταυτόχρονης αναγνώρισης και εντοπισμού αντικειμένων
(object recognition and localization - object detection) σε εικόνες.

Στοχεύει στην υλοποίηση μοντέλων CNN και στην εφαρμογή τους σε προβλήματα πραγματικού (σχεδόν)
χρόνου, όπου μία από της απαιτήσεις είναι ο γρήγορος χρόνος εκτέλεσης.

Εξετάζεται η ανάπτυξή μοντέλων CNN (για εφαρμογές αναγνώρισης και εντοπισμού αντικειμένων σε εικόνες)
σε ένα σταθερό υπολογιστικό σύστημα (PC)
καθώς και η χρήση υβριδικών ενσωματωμένων συστημάτων, τα οποία
φέρουν μονάδες CPU και GPU (όπως για παράδειγμα το Jetson TK1).

Επίσης παρουσιάζει μία σειρά από διαδικασίες βελτιστοποίησης για το
ενσωματωμένο σύστημα Jetson TK1, με στόχο την γρήγορη εκτέλεση
της διαδικασίας προς-τα-εμπρός περάσματος (forward propagation) στο εκάστοτε νευρωνικό δίκτυο,
καθώς και την μείωση της κατανάλωσης ισχύος κατά την διαδικασία εκτέλεσης.

% συνεισφορά 1: survey σχετικό με deep learning για object detection & εφαμρογή τους σε PC & K1
% συνεισφορά 2: Optimization ενός DNN για τον K1

\section{Διάρθρωση της Αναφοράς}
\label{section:layout}

Η διάρθρωση της παρούσας διπλωματικής εργασίας είναι η εξής:

\begin{itemize}
  \item{\textbf{Κεφάλαιο \ref{chapter:sota}:}
      Γίνεται ανασόπιση της ερευνητικής περιοχής που αφορά την αναγνώριση και εντοπισμό αντικειμένων με χρήση τεχνικών μηχανικής μάθησης και πιο συγκεκριμένα με χρήση μοντέλων CNN.
    }
  \item{\textbf{Κεφάλαιο \ref{chapter:theory}:} Περιγράφονται τα βασικά θεωρητικά στοιχεία
      στα οποία βασίστηκαν οι υλοποιήσεις. Γίνεται εισαγωγή στην επιστήμη
      της μηχανικής μάθησης και καταλήγει στην περιγραφή της λειτουργίας
      και χρήσης των CNN.
    }
  \item{\textbf{Κεφάλαιο \ref{chapter:tools}:} Παρουσιάζονται οι
      διάφορες τεχνικές και τα εργαλεία που χρησιμοποιήθηκαν στις
      υλοποιήσεις.
    }
  \item{\textbf{Κεφάλαιο \ref{chapter:implementations}:} Πλήρης περιγραφή των υλοποιήσεων
      διαφόρων μοντέλων CNN (AlexNet, VGG16, Tiny-YOLO).
    }
  \item{\textbf{Κεφάλαιο \ref{chapter:experiments}:} Παρουσιάζεται αναλυτικά η μεθοδολογία των
      πειραμάτων και τα αποτελέσματα.
    }
  \item{\textbf{Κεφάλαιο \ref{chapter:conclusions}:} Παρουσιάζονται τα τελικά συμπεράσματα.
    }
  \item{\textbf{Κεφάλαιο \ref{chapter:future_work}:} Αναφέρονται τα
      προβλήματα που προέκυψαν και προτείνονται θέματα για μελλοντική
      μελέτη, αλλαγές και επεκτάσεις.
    }
\end{itemize}


