\begin{center}
  \centering

  \vspace{0.5cm}
  \centering
  \textbf{\Large{Περίληψη}}
  \phantomsection
  \addcontentsline{toc}{section}{Περίληψη}

  \vspace{1cm}

\end{center}

  Η Μηχανική Μάθηση αποτελεί έναν από τους σημαντικότερους τομείς των τελευταίων
  ετών εξαιτίας της ανάπτυξης των Τεχνιτών Νευρωνικών Δικτύων.
  Εμπνευσμένα από τον τρόπο που λειτουργεί ο Εγκέφαλος και το Κεντρικό Νευρικό Σύστημα (ΚΝΣ)
  του ανθρώπου, αυτά τα υπολογιστικά μοντέλα ξεπερνούν σε απόδοση προηγούμενες μορφές Τεχνητής Νοημοσύνης
  σε διαδικασίες Μηχανικής Μάθησης.
  Τα Νευρωνικά Δίκτυα Συνέλιξης (CNNs) αποτελούν μία από τις πιο ενδιαφέρουσες μορφές αρχιτεκτονικής
  Νευρωνικών Δικτύων, τα οποία χρησιμοποιούνται για την επίλυση προβλημάτων αναγνώρισης προτύπων
  στην Μηχανική Όραση (CV). Τα CNNs ανήκουν στην κατηγορία των αλγορίθμων εκμάθησης αναπαραστάσεων, που
  σημαίνει ότι δεν απαιτείται η εξαγωγή χειρόγραφων (από τον άνθρωπο) χαρακτηριστικών για την αναγνώριση των
  προτύπων, ενισχύοντας έτσι τις μηχανές με την ικανότητα να εξάγουν μόνες τους
  τα κατάλληλα χαρακτηριστικά από δοσμένα εκ των προτέρων δεδομένα.

  Τα ενσωματωμένα συστήματα έχουν περιορισμένη υπολογιστική ικανότητα σε σύγκριση με τα desktop και server συστήματα.
  Ωστόσο είναι σχεδιασμένα για εφαρμογές με χαμηλή ενεργειακή απαίτηση, όπως
  είναι για παράδειγμα οι ρομποτικές εφαρμογές. Τα ρομπότ χρειάζεται να είναι εφοδιασμένα με την κατάλληλη 
  επεξεργαστική ισχύ που να τους επιτρέπει να πραγματοποιήσουν
  βασικές ενέργειες, όπως είναι η πλοήγηση στο χώρο,
  η χαρτογράφηση και η αναγνώριση αντικειμένων χωρίς παράλληλα να καταναλώνουν
  τόση ενέργεια που να εξαντλεί το σύστημα.

  Η παρούσα διπλωματική εργασία εστιάζει στην ανάπτυξη Νευρωνικών Δικτύων Συνέλιξης για την επίλυση θεμάτων αναγνώρισης
  και εντοπισμού αντικειμένων στο ενσωματωμένο σύστημα Jetson TK1.
  Ερευνώνται διάφορα μοντέλα CNNs όπως είναι τα AlexNet, VGG16 και YOLO. Για κάθε CNN γίνεται
  μέτρηση της μνήμης (RAM), του χρόνου εκτέλεσης και της κατανάλωσης ισχύος που απαιτείται με
  σκοπό να αξιολογηθεί η αποτελεσματικότητα τους. Τα αποτελέσματα αυτών των
  μετρήσεων συγκρίνονται με τα αντίστοιχα εξαχθέντα από εκτέλεση σε επεξεργαστή  Intel-i7-6700.

  Επιπρόσθετα, παρουσιάζεται μια σειρά από εργαλεία για την μεγιστοποίηση του λόγου της απόδοσης
  ανά μονάδα κατανάλωσης ισχύος (performance per watt) στο Jetson TK1.
  Παράλληλα, περιγράφονται οι απαραίτητες ρυθμίσεις και διαδικασίες εγκατάστασης.

  Σύμφωνα με τα αποτελέσματα, είναι ιδιαίτερα σημαντική η επιλογή κατάλληλου λογισμικού για
  την ανάπτυξη και υλοποίηση αρχιτεκτονικών CNN. Με κατάλληλη επιλογή λογισμικού και
  μια σειρά από διαδικασίες βελτιστοποίησης, η πλατφόρμα Jetson TK1 είναι ικανή να προσφέρει λύσεις
  στο πρόβλημα της αναγνώρισης και εντοπισμού αντικειμένων με χρήση CNNs και ως αυτού θεωρείται
  κατάλληλη για εφαρμογές ρομποτικής.

