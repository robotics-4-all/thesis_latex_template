\chapter{Τεχνικές Βαθιάς Μηχανικής Μάθησης και Αναγνώρισης Αντικειμένων στον χώρο}
\label{chapter:theory}

Τα τελευταία χρόνια, ο κλάδος της Τεχνητής Νοημοσύνης είναι ένας από τους πιο ραγδαία
αναπτυσσόμενους κλάδους της επιστήμης της πληροφορικής με τεράστιο
ερευνητικό και πρακτικό ενδιαφέρον.
Διαιρείται σε δύο υποκατηγορίες: την \emph{συμβολική τεχνητή
νοημοσύνη} και την \emph{υποσυμβολική τεχνητή νοημοσύνη}. Η πρώτη προσπαθεί να
επιλύσει προβλήματα τεχνητής νοημοσύνης χρησιμοποιώντας αλγοριθμικές διαδικασίες, δηλαδή
σύμβολα και λογικούς κανόνες, ενώ η δεύτερη προσπαθεί να αναπαράγει την
ανθρώπινη “ευφυΐα” μέσα από την χρήση αριθμητικών μοντέλων
που με την σύνθεσή τους προσομοιώνουν την λειτουργία του ανθρώπινου εγκεφάλου
(υπολογιστική νοημοσύνη).

Η ικανότητα ενός νοούμενου (AI) συστήματος να αποκτά από μόνο του γνώση,
εξάγοντας πρότυπα ή/και χαρακτηριστικά σημεία από τα δεδομένα,
είναι γνωστή ως \emph{Μηχανική Μάθηση} (Machine Learning - ML) \cite{michalski2013machine}.

Στο κεφάλαιο αυτό παρουσιάζονται και αναλύονται τεχνικές και αλγόριθμοι
\emph{Μηχανικής Μάθησης} με επίκεντρο τα βαθιά νευρωνικά δίκτυα (Deep Neural Networks - DNN).
Οι βασικές αρχές και
λειτουργίες των νευρωνικών αυτών δικτύων αποτελούν τις βάσεις για την περαιτέρω
μελέτη των νευρωνικών δικτύων συνέλιξης (CNN) και των εφαρμογών αυτών στο
πρόβλημα της αναγνώρισης και εντοπισμού αντικειμένων σε εικόνες.

\section{Εισαγωγή στην Επιστήμη της Σύγχρονης Μηχανικής Μάθησης}
\label{sec:theory_ml}

Η εισαγωγή του κλάδου της μηχανικής μάθησης στην επιστήμη των υπολογιστών,
επέτρεψε στους υπολογιστές να μπορούν να αντιμετωπίσουν προβλήματα αντίληψης
για τον πραγματικό κόσμο, όσο και να παίρνουν υποκειμενικές αποφάσεις.

Οι αλγόριθμοι ML, επιτρέπουν σε συστήματα AI
να προσαρμόζονται εύκολα σε καινούργια προβλήματα, απαιτώντας ελάχιστη επέμβαση από τον άνθρωπο.
Για παράδειγμα, ένα Νευρωνικό Δίκτυο που έχει εκπαιδευτεί να αναγνωρίζει γάτες σε εικόνες,
δεν απαιτεί να σχεδιαστεί και να εκπαιδευτεί από το μηδέν για να έχει την ικανότητα
να αναγνωρίζει και σκύλους.

\begin{figure}[!ht]
  \centering
  \includegraphics[width=1\textwidth]{./images/chapter3/AI_1.jpg}
  \caption[Κλάδοι και εφαρμογές της επιστήμης της Τεχνητής Νοημοσύνης]{Κλάδοι και εφαρμογές της επιστήμης της Τεχνητής Νοημοσύνης}
  \label{fig:ai_1}
\end{figure}

Πολλά προβλήματα που μέχρι πριν μερικά χρόνια λύνονταν με
“χειρόγραφη”, προγραμματισμένη από τον άνθρωπο γνώση, σήμερα επιλύονται με χρήση
αλγορίθμων ML (\autoref{fig:ai_1}). Κάποια παραδείγματα αφορούν:

\begin{itemize}
  \item{Αναγνώριση ομιλίας - Speech Recognition}
  \item{Μηχανική όραση - Computer Vision}
  \begin{itemize}
    \item{Αναγνώριση αντικειμένων σε εικόνες - Object Recognition}
    \item{Αναγνώριση και εντοπισμός της θέσης αντικειμένων σε εικόνες - Object Detection}
  \end{itemize}
  \item{Αναγνώριση ηλεκτρονικών επιθέσεων στο διαδίκτυο - Cyberattack detection}
  \item{Επεξεργασία φυσικής γλώσσας - Natural Language Processing}
  \begin{itemize}
    \item{Κατανόηση της φυσικής γλώσσας του ανθρώπου - Natural Language Understanding}
    \item{Μοντελοποίηση και χρήση της φυσικής γλώσσας του ανθρώπου από μηχανές - Natural Language Generation}
  \end{itemize}
  \item{Μηχανές αναζήτησης - Search Engines}
  \item{Αναπαράσταση γνώσης - Knowledge Representation}
  \item{Ρομποτική}
\end{itemize}

Τα προβλήματα Μηχανικής Μάθησης χωρίζονται σε τρεις μεγάλες κατηγορίες:
\begin{itemize}
  \item{Υπό επίβλεψη Μάθηση - Supervised Learning:
      Στο υπολογιστικό σύστημα δίνονται παραδείγματα εισόδου και επιθυμητής εξόδου,
      δηλαδή στα δεδομένα έχουν προηγουμένως ανατεθεί ετικέτες (labels)
      και στόχος είναι να εξαχθεί ένας γενικός κανόνας αντιστοίχισης της εισόδου στην επιθυμητή έξοδο.
      Η αναγνώριση προκαθορισμένων αντικειμένων σε εικόνες είναι ένα πρόβλημα που ανήκει 
      σε αυτή την κατηγορία.
    }
  \item{Χωρίς επίβλεψη Μάθηση - Unsupervised Learning:
      Τα δεδομένα δεν έχουν ετικέτες (labels) αφήνοντας έτσι τον αλγόριθμο ML να βρει
      από μόνος του δομές στα δεδομένα εισόδου.
    }
  \item{Εκμάθηση δια ανταμοιβής - Reinforcement Learning:
      Ο πράκτορας αλληλεπιδρά με ένα δυναμικό περιβάλλον στο οποίο πρέπει να
      εκτελέσει ένα συγκεκριμένο στόχο, χωρίς την ύπαρξη ενός “δασκάλου” που να
      ορίζει ρητά αν έχει φθάσει κοντά στον στόχο. Ένα παράδειγμα εφαρμογής
      είναι η αυτόματη πλοήγηση ενός οχήματος.
    }
\end{itemize}
Kάποια προβλήματα είναι υβριδικά, δηλαδή συνδυασμός των πιο πάνω.
Στο \autoref{fig:ml_venn_diagram} απεικονίζεται το διάγραμμα Venn των διαφόρων 
αλγοριθμικών κατηγοριών ML.
\begin{figure}[!ht]
  \centering
  \includegraphics[width=0.7\textwidth]{./images/chapter3/ml_venn_diagram.jpg}
  \caption[Διάγραμμα Venn των διαφόρων κατηγοριών μηχανικής μάθησης]{Διάγραμμα Venn των διαφόρων κατηγοριών μηχανικής μάθησης}
  \label{fig:ml_venn_diagram}
\end{figure}

Επιπλέον, οι Supervised Learning αλγόριθμοι χωρίζονται σε 2 κατηγορίες, ανάλογα
με την επιθυμητή μορφή της εξόδου του αλγόριθμου ML:
\begin{itemize}
  \item{Ταξινόμησης - Classification: Όταν η έξοδος παίρνει διακριτές τιμές (discrete).}
  \item{Regression: Όταν η έξοδος παίρνει συνεχείς τιμές.}
\end{itemize}

Γενικότερα, οι αλγόριθμοι ML ομαδοποιούνται και ανάλογα με την ομοιότητα τους
σε σχέση με την λειτουργία που εκτελούν. Πιο κάτω αναφέρονται οι πιο δημοφιλείς
αλγόριθμοι ML ομαδοποιημένοι με βάση την λειτουργία τους:
\\

\begin{minipage}{0.5\textwidth}

  \textbf{\large Regression}

  Ασχολείται με τη μοντελοποίηση της σχέσης μεταξύ των μεταβλητών που ανανεώνονται επαναληπτικά
  χρησιμοποιώντας ένα μέτρο σφάλματος για τις προβλέψεις που γίνονται από το μοντέλο
  \begin{itemize}
    \setlength\itemsep{0em}
    \item{Ordinary Least Squares Regression (OLSR)}
    \item{Linear Regression}
    \item{Logistic Regression}
    \item{Stepwise Regression}
    \item{Multivariate Adaptive Regression Splines (MARS)}
    \item{Locally Estimated Scatterplot Smoothing (LOESS)}
  \end{itemize}
\end{minipage}
\begin{minipage}{0.5\textwidth}
  \begin{center}
    \includegraphics[width=0.6\textwidth]{./images/chapter3/regression_algorithms.png}
  \end{center}
\end{minipage}

\newpage

\begin{minipage}{0.5\textwidth}

  \textbf{\large Instance-based}

  Αυτές οι μέθοδοι δημιουργούν μία βάση με
  παραδείγματα δεδομένων και συγκρίνουν τις νέες εισόδους με αυτές που έχουν
  καταχωρηθεί στην βάση δεδομένων χρησιμοποιώντας ένα μέτρο ομοιότητας,
  για την πιθανοτική εύρεση της καλύτερης αντιστοιχίας.
  \begin{itemize}
    \setlength\itemsep{0em}
    \item{k-Nearest Neighbour (kNN)}
    \item{Learning Vector Quantization (LVQ)}
    \item{Self-Organizing Map (SOM)}
    \item{Locally Weighted Learning (LWL)}
  \end{itemize}
\end{minipage}
\begin{minipage}{0.5\textwidth}
  \begin{center}
    \includegraphics[width=0.6\textwidth]{./images/chapter3/instance_based_algorithms.png}
  \end{center}
\end{minipage}

\begin{minipage}{0.5\textwidth}  %% Minipage

  \textbf{\large Regularization}

  Χρησιμοποιούνται σαν επεκτάσεις άλλων μεθόδων και
  “τιμωρούν” πολύπλοκα μοντέλα, ευνοώντας έτσι
  απλούστερα μοντέλα τα οποίο είναι συνήθως καλύτερα στην γενίκευση της
  επίλυσης του εκάστοτε προβλήματος.
  \begin{itemize}
    \setlength\itemsep{0em}
    \item{Ridge Regression}
    \item{Least Absolute Shrinkage and Selection Operator (LASSO)}
    \item{Least-Angle Regression (LARS)}
    \item{Elastic Net}
  \end{itemize}
\end{minipage}
\begin{minipage}{0.5\textwidth}
  \begin{center}
    \includegraphics[width=0.6\textwidth]{./images/chapter3/regularization_algorithms.png}
  \end{center}
\end{minipage}

\begin{minipage}{0.5\textwidth}

  \textbf{\large Dimensionality Reduction}

  Χρησιμοποιούνται για την αφαίρεση ασήμαντης πληροφορίας
  από τα δεδομένα. Πολλές από τις μεθόδους αυτές χρησιμοποιούνται σαν επεκτάσεις σε μοντέλα επίλυσης
  προβλημάτων regression και classification

  \begin{itemize}
    \setlength\itemsep{0em}
    \item{Principal Component Analysis (PCA)}
    \item{Discriminant Analysis: Linear (LDA), Mixture (MDA), Quadratic (QDA), Flexible (FDA)}
    \item{Principal Component Regression (PCR)}
    \item{Multidimensional Scaling (MDS)}
  \end{itemize}
\end{minipage}
\begin{minipage}{0.5\textwidth}
  \begin{center}
    \includegraphics[width=0.6\textwidth]{./images/chapter3/dimensional_reduction_algorithms.png}
  \end{center}
\end{minipage}

\begin{minipage}{0.5\textwidth}

  \textbf{\large Decision Trees}

  Χρησιμοποιούνται για την κατασκευή μοντέλων
  λήψης αποφάσεων, τα οποία χρησιμοποιούν τις πραγματικές τιμές των
  χαρακτηριστικών των δεδομένων.
  \begin{itemize}
    \setlength\itemsep{0em}
    \item{Classification and Regression Tree (CART)}
    \item{Conditional Decision Trees}
    \item{M5}
  \end{itemize}
\end{minipage}
\begin{minipage}{0.5\textwidth}
  \begin{center}
    \includegraphics[width=0.6\textwidth]{./images/chapter3/decision_tree_algorithms.png}
  \end{center}
\end{minipage}

\begin{minipage}{0.5\textwidth}

  \textbf{\large Bayesian}

  Εφαρμόζουν το θεώρημα του Bayes για την επίλυση τόσο προβλημάτων regression, αλλά και classification
  \begin{itemize}
    \setlength\itemsep{0em}
    \item{Naive Bayes}
    \item{Gaussian Naive Bayes}
    \item{Bayesian Network (BN)}
    \item{Bayesian Belief Network (BBN)}
  \end{itemize}
\end{minipage}
\begin{minipage}{0.5\textwidth}
  \begin{center}
    \includegraphics[width=0.6\textwidth]{./images/chapter3/bayesian_algorithms.png}
  \end{center}
\end{minipage}

\begin{minipage}{0.5\textwidth}

  \textbf{\large Clustering}

  Περιγράφουν τις κλάσεις του προβλήματος %did not like the description here...
  \begin{itemize}
    \setlength\itemsep{0em}
    \item{k-Means}
    \item{k-Medians}
    \item{Expectation Maximisation (EM)}
    \item{Hierarchical Clustering}
  \end{itemize}
\end{minipage}
\begin{minipage}{0.5\textwidth}
  \begin{center}
    \includegraphics[width=0.6\textwidth]{./images/chapter3/bayesian_algorithms.png}
  \end{center}
\end{minipage}

\begin{minipage}{0.5\textwidth}

  \textbf{\large Artificial Neural Networks (ANN)}

  Μοντέλα εμπνευσμένα από τη δομή ή/και την λειτουργία των βιολογικών νευρωνικών δικτύων.
  Χρησιμοποιούνται στην επίλυση προβλημάτων classification ή/και regression.
  \begin{itemize}
    \setlength\itemsep{0em}
    \item{Perceptron}
    \item{Back-Propagation}
    \item{Radial Basis Function Network (RBFN)}
  \end{itemize}
\end{minipage}
\begin{minipage}{0.5\textwidth}
  \begin{center}
    \includegraphics[width=0.6\textwidth]{./images/chapter3/artificial_neural_network_algorithms.png}
  \end{center}
\end{minipage}

\begin{minipage}{0.5\textwidth}

  \textbf{\large Deep Learning (DL)}

  Οι αλγόριθμοι DL είναι η σύγχρονη επέκταση των ANN, τα οποία
  εκμεταλλεύονται την αφθονία επεξεργαστικής ισχύος των σύγχρονων υπολογιστικών συστημάτων.
  \begin{itemize}
    \setlength\itemsep{0em}
    \item{Autoncoder}
    \item{Multilayer Perseptron (MLP)}
    \item{Deep Boltzmann Machine (DBM)}
    \item{Deep Belief Networks (DBN)}
    \item{Convolutional Neural Network (CNN)}
    \item{Stacked Auto-Encoders}
    \item{Recurrent Neural Networks (RNN)}
  \end{itemize}%
\end{minipage}
\begin{minipage}{0.5\textwidth}
  \begin{center}
    \includegraphics[width=0.6\textwidth]{./images/chapter3/deep_learning_algorithms.png}
  \end{center}
\end{minipage}
\\\\

Η μορφή της αναπαράστασης των δεδομένων αποτελεί σημαντικό παράγοντα στην
απόδοση των αλγορίθμων ML. Μία αναπαράσταση αποτελείται από χαρακτηριστικά (features).
Για παράδειγμα ένα χρήσιμο χαρακτηριστικό στην ταυτοποίηση ομιλητή από δεδομένα ήχου,
είναι η εκτίμηση του μεγέθους της φωνητικής έκτασης του ομιλητή.
Έτσι, πολλά προβλήματα τεχνητής νοημοσύνης μπορούν να λυθούν
με κατάλληλη σχεδίαση και επιλογή των χαρακτηριστικών για το συγκεκριμένο
πρόβλημα. Το σύνολο των χαρακτηριστικών αυτών αποτελεί την αναπαράσταση των δεδομένων
σε ένα πιο υψηλό και αφαιρετικό επίπεδο αντίληψης για τους υπολογιστές η οποία
στην συνέχεια δίνεται σαν είσοδος σε έναν απλό ML αλγόριθμο, ο οποίος έχει
μάθει να αντιστοιχεί την αναπαράσταση των δεδομένων στην επιθυμητή έξοδο.
\begin{figure}[!h]
  \centering
  \includegraphics[width=0.7\textwidth]{./images/chapter3/representation_dependency.png}
  \caption[Παράδειγμα διαφορετικών αναπαραστάσεων των δεδομένων]{Παράδειγμα διαφορετικών αναπαραστάσεων των δεδομένων}
  \label{fig:representation_dependency}
\end{figure}

Ένα απλό και κατανοητό παράδειγμα το οποίο δείχνει την εξάρτηση της επίδοσης των
αλγορίθμων ML από την μορφή της αναπαράστασης που του δίνεται, φαίνεται στο
\autoref{fig:representation_dependency}. Έστω ότι θέλουμε να
διαχωρίσουμε τα δεδομένα μας σε δύο κλάσεις, χαράζοντας μία ευθεία
μεταξύ τους. Αν αναπαραστήσουμε τα δεδομένα στο Καρτεσιανό σύστημα συντεταγμένων (αριστερό διάγραμμα),
τότε η επίλυση του προβλήματος είναι αδύνατη αφού δεν υπάρχει καμία ευθεία
που να διαχωρίζει τις δύο κλάσεις. Ωστόσο, αν αναπαραστήσουμε τα δεδομένα
στο πολικό σύστημα συντεταγμένων (δεξί διάγραμμα), τότε το πρόβλημα λύνεται
εύκολα χαράζοντας μία κάθετη ευθεία, με $r  = a, a \in [r1, r2]$.

Στα περισσότερα προβλήματα τεχνητής νοημοσύνης, η επιλογή κατάλληλων χαρακτηριστικών
είναι δύσκολο και χρονοβόρο έργο. Έστω ότι θέλουμε να αναγνωρίσουμε
πρόσωπα σε εικόνες. Ένα χαρακτηριστικό θα μπορούσε να είναι τα μάτια. Δυστυχώς όμως,
η αναγνώριση ματιών είναι ένα δύσκολο πρόβλημα αφού δεν μπορεί να
περιγραφεί πάντα επακριβώς έχοντας σαν δεδομένα τις τιμές των pixel της εικόνας.
Η γεωμετρική για παράδειγμα μορφή των ματιών σε μία εικόνα λήψης εξαρτάται από την
γωνία λήψης της εικόνας, τον φωτισμό, τις ανακλάσεις του φωτισμού,
την απόσταση από την οποία γίνεται η λήψη, την ανάλυση της κάμερας, κτλ.

Το πρόβλημα αυτό, της επιλογής κατάλληλης αναπαράστασης, μπορεί να λυθεί
χρησιμοποιώντας τεχνικές μηχανικής μάθησης για την εκμάθηση της ίδιας
της αναπαράστασης. Αυτή η προσέγγιση είναι γνωστή ως
\emph{Εκμάθηση Αναπαραστάσεων (Representation Learning)}. Οι αλγόριθμοι
εκμάθησης αναπαραστάσεων είναι ικανοί να “μάθουν” ένα καλό σετ χαρακτηριστικών (features).
\begin{figure}[!ht]
  \centering
  \includegraphics[width=0.3\textwidth]{./images/chapter3/autoencoder.png}
  \caption[Απλό μοντέλο Autoncoder με ένα κρυφό επίπεδο]{Απλό μοντέλο Autoncoder με ένα κρυφό επίπεδο.}
  \label{fig:autoencoder}
\end{figure}
Ένα απλό παράδειγμα αλγορίθμου εκμάθησης αναπαραστάσεων είναι αυτό του
Autoencoder \cite{baldi2012autoencoders} που φαίνεται στο \autoref{fig:autoencoder}.
Ο Autoencoder, στην πιο απλή του μορφή, είναι ο συνδυασμός ενός κωδικοποιητή (encoder) ο οποίος μετατρέπει
τα δεδομένα εισόδου σε μία διαφορετική αναπαράσταση, και ενός αποκωδικοποιητή
ο οποίος επαναφέρει την αναπαράσταση αυτή στην αρχική μορφή της αναπαράστασης των
δεδομένων εισόδου. Οι Autoencoders ανήκουν στην κατηγορία των Νευρωνικών
Δικτύων και είναι Unsupervised ML αλγόριθμοι.
\begin{figure}[!ht]
  \centering
  \includegraphics[width=0.5\textwidth]{./images/chapter3/cnn_deeper.jpg}
  \caption[Παράδειγμα απεικόνισης των φίλτρων ενός μοντέλου CNN για αναγνώριση προσώπου]{Παράδειγμα απεικόνισης των φίλτρων ενός μοντέλου CNN για αναγνώριση προσώπου}
  \label{fig:cnn_filter_visualization}
\end{figure}
Ένα συχνά εμφανιζόμενο πρόβλημα σε εφαρμογές τεχνητής νοημοσύνης είναι
η εύρεση και εξαγωγή χαρακτηριστικών \emph{υψηλού επιπεδου} από τα
δεδομένα. Τα μοντέλα \emph{Βαθιάς Μηχανικής Μάθησης} δίνουν λύσεις
σε αυτό το πρόβλημα εκμάθησης αναπαραστάσεων με την εισαγωγή χαρακτηριστικών
τα οποία εκφράζονται με βάση άλλες, απλούστερες αναπαραστάσεις. Αυτή η
προσέγγιση δίνει την δυνατότητα στους υπολογιστές να κατασκευάζουν σύνθετες
έννοιες χρησιμοποιώντας απλούστερες. Για παράδειγμα, η αναγνώριση αντικειμένων
μπορεί να εκφραστεί με έννοιες όπως το γεωμετρικό σχήμα των αντικειμένων,
το οποίο με την σειρά του ορίζεται από γωνίες και περιγράμματα. Επίσης,
οι γωνίες και τα περιγράμματα ορίζονται από ακμές. Στο \autoref{fig:cnn_filter_visualization},
παρουσιάζονται τα φίλτρα που έμαθε ένα μοντέλο CNN για αναγνώριση προσώπων σε εικόνες.
Το συγκεκριμένο μοντέλο CNN έχει 3 κρυφά επίπεδα (hidden layers); το πρώτο κρυφό
επίπεδο εξάγει από τα δεδομένα εισόδου (τιμές των πίξελ) πληροφορία σχετικά με
τις ακμές, το δεύτερο, έχοντας σαν είσοδο την πληροφορία παρουσίας ακμών, εξάγει πληροφορία
σχετικά με τις γωνίες και τα περιγράμματα και το τρίτο παίρνει σαν είσοδο
την πληροφορία αυτή και κατασκευάζει μοντέλα αντικειμένων, δηλαδή εξάγει πληροφορία
σχετικά με το γεωμετρικό σχήμα των αντικειμένων.

Συνοψίζοντας, η βαθιά μηχανική μάθηση, είναι μία υποκατηγορία του ML και ένας
χαρακτηριστικός αντιπρόσωπος της σύγχρονης
τεχνητής νοημοσύνης. Πιο συγκεκριμένα, είναι ένα είδος μηχανικής μάθησης, η
οποία προσδίδει σε υπολογιστικά συστήματα ευφυΐα, δηλαδή την ικανότητα
εκμάθησης με την χρήση εμπειρίας και δεδομένων. Σύμφωνα με τους
\emph{Ian Goodfellow, Yoshua Bengio και Aaron Courville}, η μηχανική μάθηση είναι η
μόνη βιώσιμη προσέγγιση στην κατασκευή συστημάτων AI τα οποία μπορούν να
αντεπεξέλθουν σε πολύπλοκα περιβάλλοντα και προβλήματα \cite{Goodfellow-et-al-2016-Book}.
Η βαθιά μηχανική μάθηση καταφέρνει να μαθαίνει να αναπαριστά τον κόσμο ως μία ένθετη
ιεραρχία εννοιών όπου η κάθε έννοια ορίζεται σε σχέση με άλλες πιο απλές έννοιες,
και πιο αφηρημένες μορφές αναπαραστάσεων σε σχέση με λιγότερο αφηρημένες.
Από το διάγραμμα Venn που βλέπουμε στο \autoref{fig:ai_venn_diagram} παρατηρούμε
ότι η βαθιά μηχανική μάθηση ανήκει στην κατηγορία της εκμάθησης αναπαραστάσεων,
η οποία με την σειρά της είναι ένα είδος μηχανικής μάθησης που χρησιμοποιείται
για την κατασκευή νοούμενων συστημάτων.

\begin{figure}[!ht]
  \centering
  \includegraphics[width=0.9\textwidth]{./images/chapter3/ai_venn_diagram.png}
  \caption[Υπερσύνολα του κλάδου της βαθιάς μηχανικής μάθησης - Διάγραμμα Venn]{%
  Διάγραμμα Venn που δείχνει πως η επιστήμη της βαθιάς μάθησης ανήκει στην κατηγορία τεχνικών εκμάθησης αναπαραστάσεων, οι οποίες τεχνικές ανήκουν με την σειρά τους στην ευρύτερη επιστήμη της μηχανικής μάθησης}
  \label{fig:ai_venn_diagram}
\end{figure}

\section{Νευρωνικά Δίκτυα με Βάθος}
\label{sec:theory_dnn}

Τα Νευρωνικά Δίκτυα είναι εμπνευσμένα από το βιολογικό νευρικό σύστημα
του ανθρώπου. Η βασική επεξεργαστική μονάδα του εγκεφάλου είναι ο \emph{νευρώνας} (\autoref{fig:neuron_bio}),
ενώ το ανθρώπινο νευρικό σύστημα αποτελείται από περίπου 86 εκατομμύρια νευρώνες και περίπου
$10^{14} - 10^{15}$ διασυνδέσεις.
\begin{figure}[!ht]
  \centering
  \includegraphics[width=0.7\textwidth]{./images/chapter3/neuron.png}
  \caption[Βιολογικός Νευρώνας]{Βιολογικός νευρώνας.}
  \label{fig:neuron_bio}
\end{figure}
Τα κύρια μέρη ενός νευρώνα είναι τα εξής:
\begin{itemize}
  \item{\textbf{Δενδρίτης (Dendrites)}: Δέχεται είσοδο από άλλους νευρώνες.}
  \item{\textbf{Σώμα του κυττάρου (Cell body)}: Εξάγει συμπεράσματα, με βάση τις εισόδους.}
  \item{\textbf{Νευράξονας (Axon)}: Συνδέει την έξοδο που λαμβάνεται από το σώμα του κυττάρου με τις νευραξoνικές απολήξεις}
  \item{\textbf{Νευραξονικές απολήξεις}: Συνδέει τον νευράξονα του εκάστοτε νευρώνα με τους τερματικούς κόμβους,
    από όπου και μεταφέρεται η πληροφορία στην είσοδο άλλων νευρώνων.}
\end{itemize}
Κάθε νευρώνας δέχεται είσοδο από άλλους νευρώνες μέσω των δενδριτών του και
στην συνέχεια επεξεργάζεται το σήμα που λαμβάνει στην είσοδο και
στέλνει το αποτέλεσμα στον νευράξονα. Τέλος, άλλοι νευρώνες συνδέονται με αυτόν
μέσω των συνάψεων του.

\begin{figure}[!ht]
  \centering
  \includegraphics[width=0.7\textwidth]{./images/chapter3/neuron_model.jpg}
  \caption[Μαθηματικό μοντέλο του νευρώνα]{Μαθηματικό μοντέλο του νευρώνα}
  \label{fig:neuron_model}
\end{figure}

Το αντίστοιχο μαθηματικό μοντέλο του νευρώνα, φαίνεται στο \autoref{fig:neuron_model}.
Η πληροφορία που μεταφέρεται από τις νευραξονικές απολήξεις ($x0$), προτού μεταφερθεί
στους δενδρίτες των επόμενων νευρώνων, αλληλεπιδρά πολλαπλασιαστικά με τις
συνάψεις ($w0*x0$). Οι παράγοντες πολλαπλασιασμού $w_n$ ονομάζονται βάρη
και αποτελούν τις μεταβλητές παραμέτρους ενός νευρώνα. Η τιμή των παραμέτρων
αυτών ελέγχει την επίδραση μεταξύ των νευρώνων. Η συνάρτηση ενεργοποίησης $f$
ελέγχει την ροή της πληροφορίας στους συνδεδεμένους νευρώνες
και προσδίδει ευελιξία και ικανότητα εκτίμησης όσον αφορά πολύπλοκες μη γραμμικές
σχέσεις στα δεδομένα εισόδου. Η έξοδος από ένα νευρώνα υπολογίζεται από τη σχέση:
\begin{gather*}
  a = f(\sum_{\imath=0}^{N}w_{\imath}x_{\imath} + b)
\end{gather*}

Η πιο απλή μορφή συνάρτησης ενεργοποίησης
είναι η σιγμοειδής συνάρτηση $\sigma(x) = 1 / (1 + e^{-x})$.
Εναλλακτικά, η σιγμοειδής συνάρτηση ενεργοποίησης μπορεί να εκφραστεί σε διακριτή μορφή ως
\[
f(x) =
  \begin{cases}
    0, & x < 0 \\
    1, & x \geq 0 \\
  \end{cases}
\]
Η επιλογή κατάλληλης συνάρτησης ενεργοποίησης των
νευρώνων δεν είναι τυχαία, αφού όπως θα δούμε στην
συνέχεια επηρεάζει την απόδοση του αλγορίθμου \emph{Backpropagation}, ο οποίος
χρησιμοποιείται για την εκπαίδευση των νευρωνικών δικτύων.

Γενικότερα, ο νευρώνας μπορεί να είναι και πολωμένος (biased - $b$) και έτσι το μαθηματικό
μοντέλο που τον περιγράφει πλήρως παίρνει την μορφή:
\begin{center}
\begin{large}
  $\sigma = f(\sum_{\imath}w_{\imath}x_{\imath} + b) = \frac{1}{1 + e^{-\sum_{\imath}w_{\imath}x_{\imath} + b}}$
\end{large}
\end{center}
Θα μπορούσαμε να ερμηνεύσουμε το αποτέλεσμα της εφαρμογής της
σιγμοειδoύς συνάρτησης ενεργοποίησης ως την πιθανότητα μίας από τις κλάσεις:
%\begin{center}
\begin{large}
\begin{gather*}
  P(y_{\imath} = 1 | x_{\imath};w) \\
  P(y_{\imath} = 0 | x_{\imath};w) = 1 - P(y_{\imath} = 1 | x_{\imath};w)
\end{gather*}
\end{large}
%\end{center}

Με εφαρμογή κατάλληλης
συνάρτησης σφάλματος στην έξοδο, ο νευρώνας έχει την συμπεριφορά ενός γραμμικού ταξινομητή
(linear classifier). Πιο συγκεκριμένα, σε περίπτωση που χρησιμοποιήσουμε την \emph{cross-entropy}
συνάρτηση σφάλματος ο νευρώνας μετατρέπεται σε δυαδικό ταξινομητή \emph{Softmax}\footnote{Ταξινομητής Softmax: \url{http://cs231n.github.io/linear-classify/\#softmax}}.

\begin{figure}[!ht]
  \centering
  \includegraphics[width=0.6\textwidth]{./images/chapter3/perceptron_and.jpg}
  \caption[Υλοποίηση πύλης AND με χρήση του μαθηματικού μοντέλου του νευρώνα]{Υλοποίηση πύλης AND με χρήση του μαθηματικού μοντέλου του νευρώνα}
  \label{fig:neuron_and}
\end{figure}

Οι τρεις θεμελιώδεις εφαρμογές του μαθηματικού μοντέλου του νευρώνα είναι η
μοντελοποίηση των λογικών πυλών \emph{AND, OR και NOT}. Στο \autoref{fig:neuron_and}
παρουσιάζεται το αντίστοιχο μοντέλο της λογικής πύλης \emph{AND}. Ο νευρώνας
δέχεται σαν είσοδο 2 σήματα ($X_{1}$,  $X_{2}$) και μία παράμετρο πόλωσης ($b=-1.5$).
Η έξοδος $a$ ορίζεται ως:
\begin{gather*}
  a = f(x) = f(X_{1}, X_{2}) =
  \begin{cases}
    0, & x < 0 \\
    1, & x \geq 0 \\
  \end{cases} \\
   x = X_{1} + X_{5} - 1.5
\end{gather*}

Η σύνδεση πολλών νευρώνων σε σε ένα γράφο δομεί ένα \emph{Νευρωνικό Δίκτυο - ΝΝ}.
Το μοντέλο ΝΝ που φαίνεται στο \autoref{fig:simple_nn}
ονομάζεται \emph{Perceptron} ή αλλιώς
\emph{Feedforward Artificial Neural Network - ANN}.
Feedforward γιατί η πληροφορία ρέει προς μία μόνο κατεύθυνση, δηλαδή
η έξοδος νευρώνα στο $\imath$ επίπεδο δεν συνδέεται με την είσοδο νευρώνα
που βρίσκεται το επίπεδο $k \leq \imath$.

Ένα ΑΝΝ έχει τα εξής χαρακτηριστικά:
\begin{itemize}
  \item{Οι διασυνδέσεις μεταξύ των νευρώνων δεν σχηματίζουν σε καμία περίπτωση κύκλο.}
  \item{Αποτελείται από 2 ή περισσότερα κρυφά επίπεδα; ένα κρυφό και το επίπεδο εξόδου}
  \item{Χρησιμοποιείται η σιγμοειδής συνάρτηση ενεργοποίησης}
\end{itemize}

\begin{figure}[!ht]
  \centering
  \includegraphics[width=0.5\textwidth]{./images/chapter3/simple_nn.jpg}
  \caption[Απλό μοντέλο NN με ένα κρυφό επίπεδο - Perceptron]{Απλό μοντέλο NN με ένα κρυφό επίπεδο - Perceptron}
  \label{fig:simple_nn}
\end{figure}

To Perceptron δεν είναι το μόνο
μοντέλο ΝΝ που ανήκει στην κατηγορία των Feedforward ANN. Όπως θα δούμε στο
\autoref{sec:theory_cnn}, τα \emph{Νευρωνικά Δίκτυα Συνέλιξης
(Convolutional Neural Networks - CNN)} ανήκουν και αυτά στην κατηγορία αυτή.

Η γενική δομή ενός νευρωνικού δικτύου φαίνεται στο \autoref{fig:multilayer_nn}.
Τα γνωρίσματα ενός τέτοιου, \emph{πολυ-επίπεδου ΝΝ}, είναι τα εξής:
\begin{itemize}
  \item{Αριθμός κρυφών επιπέδων}
  \item{Αριθμός των νευρώνων στο κάθε επίπεδο}
\end{itemize}
Η έξοδος από το εκάστοτε επίπεδο μπορεί να εκφραστεί ως
\[
  A_{\imath+1} = f_{\imath}(A_{\imath} \bullet W_{\imath} + B_{\imath})
\]
όπου ο πίνακας $A_{\imath}$ αναφέρεται στην είσοδο και έχει διαστάσεις $M \times N$,
$W$ είναι ο πίνακας με τα βάρη των νευρώνων του εκάστοτε επιπέδου με διαστάσεις
$K \times M$, και τέλος ο πίνακας $B$ διαστάσεων $K \times N$ αναφέρεται στις
τιμές πόλωσης.
Η τιμή $\imath$ αναφέρεται στον αριθμό του εκάστοτε επιπέδου του ΝΝ.
Ο αριθμός των επιπέδων, ή καλύτερα το μέγιστο μήκος του μονοπατιού που ακολουθεί
η πληροφορία από την είσοδο μέχρι την έξοδο του ΝΝ, ορίζει το \emph{βάθος} ενός ΝΝ.

\begin{figure}[!ht]
  \centering
  \includegraphics[width=0.9\textwidth]{./images/chapter3/multilayer_nn.jpg}
  \caption[Μορφή ενός πολυεπίπεδου νευρωνικού δικτύου]{Μορφή ενός πολυεπίπεδου νευρωνικού δικτύου}
  \label{fig:multilayer_nn}
\end{figure}

\begin{algorithm}[!htp]
  \caption{Αλγόριθμος Feedforward για τον υπολογισμό των εξόδων ενός επιπέδου του ΝΝ}
  \label{alg:nn_forward}
  \begin{algorithmic}[1]
    \Function{activation}{a}
      \State \Return $1.0 / (1.0 + e^{-a})$
    \EndFunction \\
    \Procedure{nn\_forward}{X, W, B, num\_layers}
      \State Starting from the input layer, use $\sigma$ to do
         a forward pass trough the network, computing the activities of the
        neurons at each layer.
      \State $k \gets 0$
      \While{$k < num\_layers$}
      \State $X^{k} \gets \Call{activation}{X \bullet W_{k} + B_{k}}$  \Comment{If we want to keep output from
        intermediate layers, we must add up one dimension on $X$}.
        \State $k \gets k+1$
      \EndWhile
    \EndProcedure
  \end{algorithmic}
\end{algorithm}

Ο \autoref{alg:nn_forward} υλοποιεί την διαδικασία υπολογισμού της εξόδου
ενός νευρωνικού δικτύου (forward pass), έχοντας σαν δεδομένα τα βάρη και τις τιμές πόλωσης
των νευρώνων του κάθε επιπέδου, καθώς και τα δεδομένα εισόδου.


\subsection{Συναρτήσεις Ενεργοποίησης}
\label{subsec:activations}

\textbf{Σιγμοειδές - Signmoid}

Η σιγμοειδής μη γραμμική συνάρτηση έχει την μορφή που περιγράφηκε στο \autoref{sec:theory_ml}.
Παίρνει σαν είσοδο έναν πραγματικό αριθμό και τον κανονικοποιεί στο διάστημα $[0, 1]$
\[
  \sigma(x) = 1 / (1 + e^{-x})
\]

\begin{figure}[!ht]
  \centering
  \includegraphics[width=0.4\textwidth]{./images/chapter3/sigmoid.jpg}
  \caption[Συνάρτηση Σιγμοειδούς συνάρτησης]{Συνάρτηση Σιγμοειδούς συνάρτησης}
  \label{fig:sigmoid}
\end{figure}

\textbf{Υπερβολική Εφαπτομένη - Tanh}

Παίρνει σαν είσοδο έναν θετικό αριθμό και τον κανονικοποιεί στο διάστημα $[-1, 1]$
χρησιμοποιώντας την πιο κάτω σχέση:
\[
  \tanh{x} = 2\sigma(2x) - 1
\]

\begin{figure}[!ht]
  \centering
  \includegraphics[width=0.4\textwidth]{./images/chapter3/tanh.jpg}
  \caption[Συνάρτηση Υπερβολικής Εφαπτωμένης]{Συνάρτηση Υπερβολικής Εφαπτομένης}
  \label{fig:tanh}
\end{figure}

\textbf{ReLU}

Μία από τις πιο δημοφιλές συναρτήσεις ενεργοποίησης τα τελευταία χρόνια.
Πρακτικά κρατά την ενεργοποίηση οριοθετημένη στο μηδέν και είναι
γρήγορη στον υπολογισμό.
\[
  f(x) = \max(0, x) \equiv f(x) =
  \begin{cases}
    x, \text{Αν} x > 0 \\
    0, \text{Διαφορετικά}
  \end{cases}
\]

\begin{figure}[!ht]
  \centering
  \includegraphics[width=0.4\textwidth]{./images/chapter3/relu.jpg}
  \caption[Συνάρτηση Rectified Linear Unit - ReLU]{Συνάρτηση Rectified Linear Unit - ReLU}
  \label{fig:relu}
\end{figure}

Το μειονέκτημά της είναι ότι κατά την διάρκεια της εκπαίδευσης του νευρωνικού
δικτύου τα βάρη ανανεώνονται με τέτοιο τρόπο που ο νευρώνας μπορεί να μην ενεργοποιηθεί
ποτέ. Αυτό έχει σαν αποτέλεσμα να “σκοτώσει” τον συγκεκριμένο νευρώνα.

\textbf{Leaky ReLU}

Η συνάρτηση ενεργοποίησης Leaky ReLU προσπαθεί να λύσει το προαναφερθέν
πρόβλημα που εμφανίζεται με την χρήση της συνάρτησης ReLU. Αντί να μηδενίζεται
για $x < 0$, η συνάρτηση Leaky ReLU έχει μικρή κλίση:
\begin{equation*}
  f(x) = 1(x<0)(ax) + 1(x\geq0)(x) =
  \begin{cases}
    x, \text{Αν} x>0 \\
    ax, \text{Διαφορετικά}
  \end{cases}
\end{equation*}

\begin{figure}[!ht]
  \centering
  \includegraphics[width=0.4\textwidth]{./images/chapter3/leaky.png}
  \caption[Συνάρτηση Leaky ReLU]{Συνάρτηση Leaky ReLU}
  \label{fig:leaky}
\end{figure}

Η τιμή της σταθεράς $a$ ορίζει την κλίση της συνάρτησης για $x<0$ και μπορεί να δοθεί
σαν παράμετρος του εκάστοτε νευρώνα.

Η πρώτη εφαρμογή της συνάρτησης
Leaky ReLU σαν συνάρτηση ενεργοποίησης νευρώνων έγινε
το 2015 από τον Kaiming He. Οι νευρώνες οι οποίοι χρησιμοποιούν την
συνάρτηση Leaky ReLU ονομάζονται νευρώνες \emph{PReLU} \cite{DBLP:journals/corr/HeZR015}.

\textbf{Maxout}

Η συνάρτηση ενεργοποίησης Maxout \cite{goodfellow2013maxout} είναι η γενίκευση της συνάρτησης Leaky ReLU.
Ένας νευρώνας Maxout, υπολογίζει την συνάρτηση
\begin{equation*}
  f(x) = max(w_{1}x+b_{2}, w_{2}x+b_{2})
\end{equation*}
Η πιο πάνω συνάρτηση έχει τέσσερις παραμέτρους
($w_{1}$, $w_{2}$, $b_{1}$ και $b_{2}$). Επίσης, οι συναρτήσεις
ReLU και Leaky ReLU είναι ειδικές περιπτώσεις της συνάρτησης Maxout.
Για παράδειγμα, για $w_{1}, b_{1} = 0$ παίρνει την μορφή της συνάρτησης ReLU.
Το μειονέκτημα αυτής της συνάρτησης ενεργοποίησης, σε σχέση με την συνάρτηση
ReLU, είναι ότι διπλασιάζει τις παραμέτρους κάθε νευρώνα.

\begin{figure}[!ht]
  \centering
  \includegraphics[width=1\textwidth]{./images/chapter3/maxout.png}
  \caption[Συνάρτηση Maxout]{Συνάρτηση Maxout}
  \label{fig:maxout}
\end{figure}

Πρακτικά, τα μοντέλα ΝΝ που χρησιμοποιούν την συνάρτηση Maxout έχουν την
ικανότητα να μάθουν, πέρα από την συσχέτιση μεταξύ των κρυμμένων επιπέδων,
και την συνάρτηση ενεργοποίησης όλων των νευρώνων ενός ή περισσότερων
επιπέδων.
Στο \autoref{fig:maxout} φαίνονται διάφορες μορφές της συνάρτησης Maxout,
μετά από την εκπαίδευση δικτύων Maxout.

\subsection{Συναρτήσεις Σφάλματος/Κόστους}

Από τις προηγούμενες παραγράφους γίνεται κατανοητό ότι τα νευρωνικά δίκτυα χρησιμοποιούνται για την επίλυση
προβλημάτων classification και regression. Γενικότερος στόχος είναι να παρθεί
μία απόφαση, η οποία φαίνεται στην έξοδο του μοντέλου, έχοντας σαν είσοδο
τα δεδομένα του εκάστοτε προβλήματος.
Έτσι προκύπτει ότι στα μοντέλα ΝΝ τα οποία χρησιμοποιούνται για την επίλυση
τέτοιων προβλημάτων, ενσωματώνεται ένας ταξινομητής (classifier). Ο ταξινομητής
αυτός συνήθως αποτελεί το τελευταίο επίπεδο ενός ΝΝ.

Οι συναρτήσεις σφάλματος έχουν στόχο να τιμωρήσουν τις λανθασμένες αποφάσεις
που λαμβάνονται από τον ταξινομητή, και άρα στην έξοδο ενός ΝΝ,
κατά την διάρκεια της εκπαίδευσης του.

Μία συνάρτηση σφάλματος έχει την γενική μορφή:
\[
  E_{total} = f(target - output)
\]

Σε προβλήματα κατηγοριοποίησης (classification), όπως για παράδειγμα
η αναγνώριση αντικειμένων σε εικόνες, η πιο γνωστή και συχνά χρησιμοποιούμενη συνάρτηση ταξινόμησης
είναι η συνάρτηση \emph{Softmax}, η οποία έχει την μορφή:
\begin{equation*}
  f_{j}(z) = \frac{e^{z_{j}}}{\sum_{k\not=j}e^{z_{k}}} = P(y=\jmath|x)
\end{equation*}
όπου ο δείκτης $j$ αναφέρεται σε ένα στοιχείο του διανύσματος των πιθανών κλάσεων.
Η πιο πάνω μαθηματική εξίσωση μας δίνει την πιθανότητα
η έξοδος $y$ να ανήκει σε μία εκ του συνόλου των πιθανών κλάσεων, έχοντας
υπό συνθήκη τα δεδομένα εισόδου $x$.
Πιο συγκεκριμένα, παίρνει σαν είσοδο ένα διάνυσμα πραγματικών τιμών ($z$)
και “κατασκευάζει” ένα καινούργιο διάνυσμα του οποίου τα στοιχεία είναι
κανονικοποιημένα στο διάστημα $[0, 1]$ και το άθροισμα τους ισούται με την μονάδα,
δηλαδή $\sum_{k=1}^{N} f_{\jmath}(z) = 1$.

Η συνάρτηση σφάλματος που χρησιμοποιείται στην περίπτωση του ταξινομητή Softmax 
είναι η συνάρτηση \emph{cross-entropy}:
\begin{equation*}
  L_{\imath} = - \log{(\frac{e^{f_{y_{\imath}}}}{\sum_{\jmath}^{} e^{f_{\jmath}}})}
\end{equation*}


Σε προβλήματα regression οι πιο συχνά χρησιμοποιούμενες συναρτήσεις
σφάλματος είναι οι \emph{L1 Norm} και \emph{L2 Norm}.
\begin{gather*}
  L1_{\imath} = \abs{f - y_{\imath}}_{1} \\
  L2_{\imath} = \abs{f - y_{\imath}}^{2}
\end{gather*}
Η πρώτη είναι η συνάρτηση μέσης τιμής ενώ η δεύτερη
είναι η συνάρτηση του τετραγώνου της μέσης τιμής του σφάλματος.

\subsection{Αλγόριθμος Backpropagation}

Ο αλγόριθμος \emph{backpropagation} (\autoref{alg:backpropagation}) εμφανίστηκε το 1970 και υποτιμήθηκε
μέχρι το 1986, όταν και οι David Rumelhart, Geoffrey Hinton, και Ronald Williams
απέδειξαν την αποδοτικότητα του στην εκπαίδευση των νευρωνικών δικτύων, κυρίως
όσον αφορά στην ταχύτητα \cite{rumelhart1988learning}.
Σήμερα ο αλγόριθμος backpropagation χρησιμοποιείται κατά κόρον για την εκπαίδευση
μεγάλων νευρωνικών δικτύων με εκατομμύρια παραμέτρους.

Σκοπός του αλγόριθμου \emph{backpropagation} είναι να
ελαχιστοποιήσει το σφάλμα, δοσμένης μίας συνάρτησης σφάλματος, ορισμένη στον χώρο
των βαρών $w$, χρησιμοποιώντας τον αλγόριθμο \emph{Gradient Descent}.
Υπολογίζει δε το σφάλμα και ανανεώνει ανάλογα τις τιμές του
πολυεπίπεδου νευρωνικού δικτύου, ακολουθώντας μία προς τα πίσω διαδικασία.

Πρακτικά, ο αλγόριθμος backpropagation εφαρμόζει τον κανόνα
της αλυσιδωτής παραγώγισης (gradient chain rule) για τον υπολογισμό των
μερικών παραγώγων, δοσμένης μίας συνάρτησης σφάλματος.

\begin{equation*}
  \frac{\partial z}{\partial x} = \frac{\partial z}{\partial y}\bullet\frac{\partial y}{\partial x}
\end{equation*}

Το πρώτο βήμα για την ελαχιστοποίηση του σφάλματος είναι ο υπολογισμός
της παραγώγου της συνάρτησης σφάλματος ως προς τις παραμέτρους $w$ κάθε επιπέδου,
ή καλύτερα κάθε νευρώνα, του νευρωνικού δικτύου.
\begin{equation*}
  \frac{\partial E_{total}}{\partial w_{ij}}
\end{equation*}

Στο \autoref{fig:backprop} φαίνεται η διαδικασία υπολογισμού της παραγώγου της συνάρτησης
σφάλματος ως προς τις παραμέτρους $w$
ενός νευρωνικού δικτύου που αποτελείται από 1 κρυφό επίπεδο και κάθε επίπεδο
αποτελείται από δύο νευρώνες.

\begin{figure}[!ht]
  \centering
  \hspace*{2cm}
  \includegraphics[width=0.6\textwidth]{./images/chapter3/backprop.png}
  \caption[Διαδικασία Backward propagation]{Διαδικασία Backward propagation}
  \label{fig:backprop}
\end{figure}

Η εξίσωση μερικής παραγώγισης της συνάρτησης σφάλματος ως προς την παράμετρο
$w_{1}$ μπορεί να γραφεί σε πιο αναλυτική μορφή:
\begin{gather*}
  \frac{\partial E_{total}}{\partial w_{1}} = (\sum_{k=1}^{O}\frac{\partial E_{total}}{\partial out_{k}}*\frac{\partial out_{k}}{\partial net_{k}}*\frac{\partial net_{k}}{\partial out_{h1}})*\frac{\partial out_{h1}}{\partial net_{h1}}*\frac{\partial net_{h1}}{\partial w_{1}} \\
  \frac{\partial E_{total}}{\partial w_{1}} = (\sum_{k=1}^{O}\delta_{ho}*w_{ho}) * out_{h1}(1-out_{h1}) * i_{1} \\
  \frac{\partial E_{total}}{\partial w_{1}} = \delta_{h1}i_{1}
\end{gather*}
Η αλυσιδωτή παραγώγιση εμπλέκει και την μερική παραγώγιση της
συνάρτησης ενεργοποίησης κάθε νευρώνα ως προς τις παραμέτρους $w$ του.
Έτσι, η επιλογή της
συνάρτησης ενεργοποίησης επηρεάζει και την απόδοση, τόσο σε χρόνο εκτέλεσης,
όσο και σε σφάλματα λόγω παραγώγισης, του αλγορίθμου backpropagation.

Η δε ανανέωση των βαρών $w$ γίνεται με βάση την σχέση
\begin{equation*}
  w_{i}^{+} = w_{i} - step * \frac{\partial E_{total}}{\partial w_{i}}
\end{equation*}
στην οποία αφαιρούνται από την αρχική τιμή η μεταβολή του σφάλματος πολλαπλασιασμένη
με ένα βαθμωτό μέγεθος. Το μέγεθος αυτό οποίο ονομάζεται \emph{ρυθμός ανανέωσης}
των βαρών και είναι σταθερά της διαδικασίας εκπαίδευσης.

\makeatletter
\newcommand{\HEADER}[1]{\State\underline{\textsc{#1}}}
  \newcommand{\ENDHEADER}{}
\makeatother
\newcommand{\STATEI}[1]{\State
  \begin{tabular}{@{}p{\dimexpr \textwidth-\labelwidth}@{}}%
    \hangindent \algorithmicindent
    \hangafter 1
    #1
  \end{tabular}
}

\begin{algorithm}[H]
  \caption{Backpropagation learning algorithm}
  \label{alg:backpropagation}
  \begin{algorithmic}
  \For{d in data}
    \HEADER{Forwards Pass}
    \STATEI{Starting from the input layer, do a forward pass trough the network,}
    \STATEI {computing the activities of the neurons at each layer.}
    \ENDHEADER
    \HEADER{Backwards Pass}
    \STATEI{Compute the derivatives of the error function with respect to}
    \STATEI{the output layer activities}
      \For{layer in layers}
      \STATEI{Compute the derivatives of the error function with respect to}
      \STATEI{the inputs of the upper layer neurons}
      \STATEI{Compute the derivatives of the error function with respect to}
      \STATEI{the weights between the outer layer and the layer below}
      \STATEI{Compute the derivatives of the error function with respect}
      \STATEI{to the activities of the layer below}
      \EndFor
      \STATEI{Updates the weights.}
    \ENDHEADER
  \EndFor
  \end{algorithmic}
\end{algorithm}

Η πλήρης ανάλυση του αλγόριθμου backpropagation ξεφεύγει από τα πλαίσια
της παρούσας διπλωματικής εργασίας γιατί δεν ασχολείται με την εκπαίδευση
των νευρωνικών δικτύων.

\section{Νευρωνικά Δίκτυα Συνέλιξης}
\label{sec:theory_cnn}

Έως τώρα έγινε αναφορά στα πολυεπίπεδα νευρωνικά δίκτυα και την γενικότερη
λειτουργία τους. Σε αυτή την παράγραφo γίνεται αναφορά σε συγκεκριμένα μοντέλα
πολυεπίπεδων νευρωνικών δικτύων και πιο συγκεκριμένα για τα
\emph{Νευρωνικά Δίκτυα Συνέλιξης}. Τα συγκεκριμένα μοντέλα χρησιμοποιούνται
σήμερα κυρίως στα προβλήματα της αναγνώρισης και εντοπισμού αντικειμένων
σε εικόνες.

Ο τρόπος λειτουργίας τους είναι όμοιος με αυτόν που παρουσιάστηκε στο % το ; είναι ελληνικό ερωτηματικό, στα αγγλικά είναι άνω τελεία.
\autoref{sec:theory_dnn}. Δηλαδή, αποτελούνται από πολλά επίπεδα, όπου το κάθε επίπεδο αποτελείται
από έναν αριθμό νευρώνων οι οποίοι έχουν σαν παραμέτρους εκμάθησης τα βάρη τους ($w_{\jmath}^{\imath}$)
και την τιμή πόλωσης ($b^{\imath}$).
Κάθε νευρώνας δέχεται ένα σήμα εισόδου, εφαρμόζει μία πράξη εσωτερικού γινομένου σε αυτό,
και προαιρετικά εφαρμόζει στο αποτέλεσμα μία μη γραμμική συνάρτηση.
Το τελευταίο επίπεδο των CNN είναι πλήρες συνδεδεμένο και έχει μία
συνάρτηση σφάλματος.
Η διαφορά των μοντέλων CNN από τα κλασσικά ANN είναι ότι τα δεδομένα εισόδου
είναι εικόνες.

Τα CNN μοντελοποιούν μικρά
τμήματα πληροφορίας, τα οποία στην συνέχεια ενώνονται για να δημιουργήσουν
υψηλότερου επιπέδου πληροφορία. Για παράδειγμα σε ένα
μοντέλο CNN το πρώτο επίπεδο προσπαθεί να εντοπίσει ακμές, το δεύτερο
επίπεδο παίρνει την πληροφορία των ακμών και προσπαθεί να εντοπίσει περιγράμματα,
κτλ.

Σε κάθε εικονοστοιχείο της εικόνας αντιστοιχούν 3 τιμές (RGB) και άρα η είσοδος σε ένα
CNN έχει τρεις διαστάσεις όπως φαίνεται και στο \autoref{fig:cnn_1}.
Για παράδειγμα ένα CNN το οποίο έχει σχεδιαστεί να δέχεται σαν είσοδο εικόνες ανάλυσης $80\times60$
έχει επίπεδο εισόδου διαστάσεων $80\times60\times3$.

\begin{figure}[!ht]
  \centering
  \includegraphics[width=0.8\textwidth]{./images/chapter3/cnn.jpg}
  \caption[Τρισδιάστατη κατανομή των νευρώνων στα CNN]{Τρισδιάστατη κατανομή των νευρώνων στα CNN}
  \label{fig:cnn_1}
\end{figure}

Κάθε επίπεδο ενός CNN παίρνει σαν είσοδο μία μορφή όγκου την οποία
και μετασχηματίζει σε μία άλλη μορφή όγκου.

Οι τρεις βασικοί τύποι επιπέδων που χρησιμοποιούνται σε αρχιτεκτονικές CNN είναι:
\begin{itemize}
  \item{Επίπεδο Συνέλιξης - Convolutional Layer (CONV)}
  \item{Επίπεδο Υπόδειγματοληψίας- Pooling Layer (POOL)}
  \item{Πλήρως Συνδεδεμένο Επίπεδο - Fully-Connected Layer (FC)}
\end{itemize}
Τα επίπεδα CONV και FC έχουν παραμέτρους, δηλαδή
βάρη και τιμή πόλωσης των νευρώνων, ενώ τα επίπεδα POOL εκτελούν λειτουργία
δειγματοληψίας στα δεδομένα εισόδου τους.

\subsection{Επίπεδο Συνέλιξης}

Τα επίπεδα συνέλιξης είναι ο πυρήνας των μοντέλων CNN. Οι παράμετροι ενός
επιπέδου CONV είναι μία σειρά από δισδιάστατα φίλτρα τα οποία όμως εκτείνονται
σε όλο το σε όλο το βάθος του όγκου εισόδου. Το βάθος των φίλτρων αυτών
ισούται με το βάθος του όγκου στην είσοδο.

Όπως προαναφέρθηκε, τα επίπεδα CONV εφαρμόζουν πράξη συνέλιξης πάνω στα
δεδομένα εισόδου. Αυτό επηρρεάζει την δομή των "τοπικών" διασυνδέσεων.
Στο παράδειγμα του σχήματος \ref{fig:cnn_2} βλέπουμε πως ο κάθε νευρώνας
του επιπέδου συνέλιξης συνδέεται με μία περιοχή του όγκου στην είσοδό του.

\begin{figure}[!ht]
  \centering
  \includegraphics[width=0.4\textwidth]{./images/chapter3/cnn_2.jpg}
  \caption[%
    Παράδειγμα διασύνδεσης τρισδιάστατης εισόδου με την τρισδιάστατη δομή των
    νευρώνων ενός επιπέδου συνέλιξης (CONV)]{%
    Παράδειγμα διασύνδεσης τρισδιάστατης εισόδου με την τρισδιάστατη δομή των
    νευρώνων ενός επιπέδου συνέλιξης (CONV)}
  \label{fig:cnn_2}
\end{figure}

Η συνέλιξη ενός φίλτρου με τον τον όγκο εισόδου παράγει έναν \emph{χάρτη ενεργοποίησης (activation map)},
με τον τρόπο που φαίνεται στο \autoref{fig:cnn_activation_map}. Στο παράδειγμα αυτό
εφαρμόζεται φίλτρο διαστάσεων $5\times5\times3$ σε έναν όγκο $32 \times 32 \times 3$ και παράγεται
ένας χάρτης ενεργοποίησης διαστάσεων $28\times28\times1$.

\begin{figure}[!ht]
  \centering
  \includegraphics[width=1\textwidth]{./images/chapter3/cnn_activation_map.png}
  \caption[Συνέλιξη φίλτρου ενός επιπέδου συνέλιξης με τον όγκο εισόδου και παραγωγή ενός χάρτη ενεργοποίησης]{Συνέλιξη φίλτρου ενός επιπέδου συνέλιξης με τον όγκο εισόδου και παραγωγή ενός χάρτη ενεργοποίησης}
  \label{fig:cnn_activation_map}
\end{figure}

Η μείωση των διαστάσεων μήκους και πλάτους από $32 \times 32$ σε $28 \times 28$ οφείλεται στον τρόπο με τον οποίο
εκτελείται η πράξη της συνέλιξης των φίλτρων με τον όγκο εισόδου  (\autoref{fig:cnn_conv}).
Οι διαστάσεις του όγκου εξόδου, έχοντας σαν είσοδο όγκο διαστάσεων $N \times N \times d$ και φίλτρων $F \times F \times d$ υπολογίζονται, στην απλούστερη περίπτωση με βάση την σχέση
\begin{equation*}
  outsize = (N-F) + 1
\end{equation*}


\begin{figure}[!ht]
  \centering
  \includegraphics[width=1\textwidth]{./images/chapter3/cnn_conv.png}
  \caption[Διαδικασία Συνέλιξης]{Διαδικασία Συνέλιξης}
  \label{fig:cnn_conv}
\end{figure}

Η τιμή του βάθους του όγκου στην έξοδο ενός επιπέδου CONV αντιστοιχεί στον αριθμό των φίλτρων που
εφαρμόζονται στον όγκο εισόδου. Δηλαδή ο αριθμός των χαρτών ενεργοποίησης
αντιστοιχεί στον αριθμό των φίλτρων. Αν για παράδειγμα ο όγκος εισόδου είναι
διαστάσεων $32 \times 32 \times 3$ και εφαρμοστούν δέκα φίλτρα συνέλιξης διαστάσεων $5\times5\times3$,
ο όγκος εξόδου θα είναι διαστάσεων $28\times28\times3$ \autoref{fig:cnn_num_filters}.
Ο αριθμός των φίλτρων είναι μία παράμετρος, ή καλύτερα \emph{υπερ-παράμετρος} των επιπέδων συνέλιξης.

\begin{figure}[!ht]
  \centering
  \includegraphics[width=1\textwidth]{./images/chapter3/cnn_num_filters.png}
  \caption[Αντιστοιχία του αριθμού των φίλτρων ενός επιπέδου συνέλιξης με το βάθος του όγκου στην έξοδο]{Αντιστοιχία του αριθμού των φίλτρων ενός επιπέδου συνέλιξης με το βάθος του όγκου στην έξοδο}
  \label{fig:cnn_num_filters}
\end{figure}

Ωστόσο, το βήμα μετατόπισης (stride) του φίλτρου πάνω στην είσοδο είναι και αυτό
μία υπέρ-παράμετρος (hyperparameter) των επιπέδων συνέλιξης.
Χρησιμοποιώντας βήμα μετατόπισης (S) διάφορο της μονάδας καταλήγουμε στην πιο κάτω
εξίσωση για τον υπολογισμό του όγκου εξόδου:
\begin{equation*}
  outsize = (N-F)/S + 1
\end{equation*}

Ένα πρόβλημα που εμφανίζεται στην περίπτωση των μοντέλων CNN με μεγάλο αριθμό
κρυφών επιπέδων είναι η γρήγορη μείωση των διαστάσεων μήκους και πλάτους του
όγκου, το οποίο είναι αποτέλεσμα της διαδοχικής εφαρμογής πράξεων συνέλιξης (\autoref{fig:cnn_shrunk}).

\begin{figure}[!ht]
  \centering
  \includegraphics[width=1\textwidth]{./images/chapter3/cnn_shrunk.png}
  \caption[Διαδοχικές εφαρμογές του τελεστή συνέλιξης προκαλούν μείωση των διαστάσεων μήκους και πλάτους του όγκου]{Διαδοχικές εφαρμογές του τελεστή συνέλιξης προκαλούν μείωση των διαστάσεων μήκους και πλάτους του όγκου}
  \label{fig:cnn_shrunk}
\end{figure}

Αυτή η συμπεριφορά είναι ανεπιθύμητη αφού περιορίζει και τις διαστάσεις των φίλτρων
που μπορούμε να χρησιμοποιήσουμε σε κάθε επίπεδο CONV. Η χρήση φίλτρων μεγάλων
διαστάσεων φέρει σαν αποτέλεσμα την γρήγορη μείωση των διαστάσεων του όγκου.

Για να αποτρέψουμε αυτή την συμπεριφορά μπορούμε να επεκτείνουμε τις διαστάσεις
μήκους και πλάτους, προσθέτοντας μηδενικά στα σύνορα του όγκου εισόδου του
εκάστοτε επιπέδου CONV. Η διαδικασία αυτή ονομάζεται
\emph{zero-padding} και φαίνεται στο \autoref{fig:cnn_zero_padding}.
\begin{figure}[!ht]
  \centering
  \includegraphics[width=0.6\textwidth]{./images/chapter3/cnn_zero_padding.png}
  \caption[Zero Padding]{Zero Padding}
  \label{fig:cnn_zero_padding}
\end{figure}
Το μέγεθος του συνόρου που προστίθεται είναι η τρίτη υπέρ-παράμετρος ενός
επιπέδου συνέλιξης.

Με την εισαγωγή της υπέρ-παραμέτρου zero-padding η εξίσωση υπολογισμού του όγκου
εξόδου έχει την μορφή:
\begin{equation*}
  outsize = (N - F + 2P)/S + 1
\end{equation*}

Συνοψίζοντας, ένα επίπεδο συνέλιξης έχει τα εξής χαρακτηριστικά:
\begin{itemize}
  \item{Διαστάσεις όγκου εισόδου: $W_{1} \times H_{1} \times D_{1}$}
  \item{Hyperparameters:}
    \begin{itemize}
      \item{K: Αριθμός φίλτρων}
      \item{F: Μέγεθος του φίλτρου ($F \times F$)}
      \item{S: Βήμα μετατόπισης}
      \item{P: Ποσότητα zero-padding}
    \end{itemize}
  \item{Διαστάσεις όγκου εξόδου: $W_{2} \times H_{2} \times D_{2}$, $D_{2} = K$} όπου:
    \begin{itemize}
      \item{$W_{2} = (W_{1} - F)/S + 1$}
      \item{$H_{2} = (H_{1} - F)/S + 1$}
      \item{$D_{2} = D_{1}$}
    \end{itemize}
\end{itemize}


\subsection{Επίπεδο Υπό-δειγματοληψίας - Pooling layer}

Συνήθως τα επίπεδα υπό-δειγματοληψίας προστίθενται στο δίκτυο, μεταξύ διαδοχικών
επιπέδων συνέλιξης. Η λειτουργία τους είναι να μειώσουν τις χωρικές
διαστάσεις των αναπαραστάσεων, μειώνοντας έτσι τον αριθμό των
παραμέτρων και άρα τους υπολογισμούς που γίνονται στο νευρωνικό δίκτυο, ενεργώντας σαν
μία συνάρτηση υποδειγματοληψίας
(\autoref{fig:cnn_pool}).

\begin{figure}[!ht]
  \centering
  \includegraphics[width=0.6\textwidth]{./images/chapter3/cnn_pool.jpg}
  \caption[Επίπεδο Υποδειγματοληψίας - Pooling layer]{Επίπεδο Υπό-δειγματοληψίας - Pooling layer}
  \label{fig:cnn_pool}
\end{figure}
Πιθανές συναρτήσεις υπό-δειγματοληψίας είναι οι συναρτήσεις \emph{max, average και L2-Norm}

Στο \autoref{fig:cnn_pool_max} βλέπουμε το αποτέλεσμα της εφαρμογής της συνάρτησης
δειγματοληψίας $max(\vec{v})$ πάνω σε ένα πλέγμα διαστάσεων $4 \times 4$.

\begin{figure}[!ht]
  \centering
  \includegraphics[width=0.6\textwidth]{./images/chapter3/cnn_pool_max.png}
  \caption[Συνάρτηση υπό-δειγματοληψίας Max - Max Pooling]{Συνάρτησης υπό-δειγματοληψίας Max - Max Pooling}
  \label{fig:cnn_pool_max}
\end{figure}

Τα χαρακτηριστικά των συναρτήσεων υπό-δειγματοληψίας είναι:
\begin{itemize}
  \item{Διαστάσεις όγκου εισόδου: $W_{1} \times H_{1} \times D_{1}$}
  \item{Hyperparameters:}
    \begin{itemize}
      \item{F: Χωρική τους έκταση ($F \times F$)}
      \item{S: Βήμα μετατόπισης}
    \end{itemize}
  \item{Διαστάσεις όγκου εξόδου: $W_{2} \times H_{2} \times D_{2}$, $D_{2} = K$} όπου:
    \begin{itemize}
      \item{$W_{2} = (W_{1} - F)/S + 1$}
      \item{$H_{2} = (H_{1} - F)/S + 1$}
      \item{$D_{2} = D_{1}$}
    \end{itemize}
\end{itemize}


\subsection{Πλήρως Συνδεδεμένο Επίπεδο - Fully-connected layer}

Ένα πλήρως συνδεδεμένο επίπεδο συνδέεται με όλους τους νευρώνες στο
προηγούμενο επίπεδο, όπως γίνεται στα απλά μοντέλα NN (Πολυεπίπεδος Perceptron),

Συνήθως το τελευταίο επίπεδο σε ένα CNN είναι πλήρως συνδεδεμένο και πιο
συγκεκριμένα έχει τόσους νευρώνες όσες και οι κλάσεις της πρόβλεψης. Για
παράδειγμα, ένα CNN που χρησιμοποιείται για αναγνώριση αντικειμένων σε
εικόνες CIFAR-10 έχει το τελευταίο επίπεδο του πλήρως συνδεδεμένο και %what is CIFAR-10? (ektos an to exeis eksigisei pio prin)
αποτελείται από 10 νευρώνες.

\section{Επισκόπηση Τεχνικών Βαθιάς Μηχανικής Μάθησης στην Αναγνώριση και Εντοπισμό Αντικειμένων}
\label{sec:theory_sota}

Τόσο η αναγνώριση αντικειμένων (object recognition) όσο και ο εντοπισμός
της θέσης των αντικειμένων αυτών (detection/localization) σε εικόνες,
είναι μία ερευνητική περιοχή με τεράστιο ενδιαφέρον,
η οποία απασχολεί πληθώρα ερευνητών. Η επιστήμη της Μηχανικής Όρασης (Computer Vision) %ML?? Machine learning?
στοχεύει στο να δώσει λύσεις στα συγκεκριμένα προβλήματα, εισάγοντας αναλυτικά
ή και πιθανοτικά μαθηματικά μοντέλα.

Ο κλάδος της Βαθιάς Μηχανικής Μάθησης (Deep Learning - DL) \cite{Goodfellow-et-al-2016-Book}
ανάγει το πρόβλημα της εύρεσης χαρακτηριστικών σημείων για την αναγνώριση αντικειμένων,
στην εκμάθηση αναπαραστάσεων \cite{bengio2013representation}
με την χρήση Νευρωνικών Δικτύων Συνέλιξης. %έχεις δώσει το ακρωνύμιο και πιο πάνω
Οι πρώτες εφαρμογές Νευρωνικών Δικτύων Συνέλιξης, για την αναγνώριση αντικειμένων
σε εικόνες, αναπτύχθηκαν το 1990 από τον Yann LeCun.
Η πιο γνωστή και επιτυχής είναι το δίκτυο LeNet \cite{lecun1998gradient}, το οποίο
χρησιμοποιήθηκε για την αναγνώριση ψηφίων σε εικόνες.
Ωστόσο, η εισαγωγή των CNN στον κλάδο της Μηχανικής Όρασης έγινε το 2012 με
την ανάπτυξη του δικτύου AlexNet \cite{NIPS2012_4824}, από τους Alex Krizhevsky,
Ilya Sutskever και Geoffrey E. Hinton. To δίκτυο AlexNet χρησιμοποιήθηκε
στον διαγωνισμό ImageNet ILSVRC challenge, το 2012, κερδίζοντας με διαφορά
10,9\% στο σφάλμα αναγνώρισης αντικειμένων σε σύνολο 1000 κλάσεων.
Με την εμφάνιση του δικτύου AlexNet, η ερευνητική κοινότητα ξεκίνησε να
πιστεύει στην αποτελεσματικότητα των Νευρωνικών Δικτύων Συνέλιξης σε εφαρμογές αναγνώρισης
αντικειμένων σε εικόνες. Συνέχεια στο έργο του Alex Krizhevsky δόθηκε το 2013,
αναπτύσσοντας το ZF-Net \cite{DBLP:journals/corr/ZeilerF13} το οποίο είναι
βασισμένο στην αρχιτεκτονική του δικτύου AlexNet.

%Μέχρι σήμερα, έχουν σχεδιαστεί
%και αναπτυχθεί διάφορα μοντέλα Νευρωνικών Δικτύων Συνέλιξης για
%αναγνώριση αντικειμένων, με πιο πρόσφατο το ResNet ,
%το οποίο αναπτύχθηκε από τον Kaiming He \cite{DBLP:journals/corr/HeZRS15}.
%Το ResNet (Residual Network) έχει την
%ιδιαιτερότητα απουσίας πλήρως συνδεδεμένων επιπέδων και είναι από τα πιο δημοφιλή
%μοντέλα που εφαρμόζονται σε πρακτικά προβλήματα αναγνώρισης αντικειμένων σε
%εικόνες \cite{DBLP:journals/corr/HeZR016}.

Τα προαναφερθέντα μοντέλα Νευρωνικών Δικτύων Συνέλιξης δίνουν
λύσεις μόνο στο πρόβλημα της αναγνώρισης και όχι
του εντοπισμού της θέσης των αντικειμένων.
Το 2013, ερευνητές εργαζόμενοι στην Google Inc. σχεδίασαν και υλοποίησαν ένα
μοντέλο CNN το οποίο δίνει λύση στο πρόβλημα της ταυτόχρονης
αναγνώρισης και εντοπισμού αντικειμένων σε εικόνες.
Το μοντέλο αυτό, που φέρει το όνομα \emph{DetectorNet}, είναι ομαδική εργασία των
Christian Szegedy, Alexander Toshev και Dumitru Erhan \cite{szegedy2013deep}.
Ωστόσο οι τεράστιες απαιτήσεις του DetectorNet σε υπολογιστικούς πόρους αποτελεί
ένα από τα μειονεκτήματά του που το καθιστούν ακατάλληλο για εφαρμογή σε
προβλήματα σχεδόν πραγματικού χρόνου όπως για παράδειγμα σε ένα
ρομποτικό σύστημα.

Την ίδια χρονιά (2013), η ερευνητική ομάδα του πανεπιστημίου New York \\
University
(Pierre Sermanet, David Eigen, Xiang Zhang, Michael Mathieu, Rob Fergus, Yann LeCun)
εμφανίστηκε στον διαγωνισμό ILSVRC challenge 2013 με ένα μοντέλο CNN το οποίο
έχει την μορφή ενός απλού δικτύου με την ικανότητα εκμάθησης
των πλαισίων (bounding boxes) στα οποία ανήκουν τα αντικείμενα που αναγνωρίζονται.
Το μοντέλο φέρει το όνομα \emph{OverFeat} \cite{sermanet2013overfeat} και ήταν
ο νικητής του συγκεκριμένου διαγωνισμού στην κατηγορία εντοπισμού αντικειμένων.

Τα πιο πρόσφατα και πλέον δημοφιλή νευρωνικά δίκτυα για ταυτόχρονη
αναγνώριση και εντοπισμό αντικειμένων σε εικόνες είναι τα μοντέλα
ResNet \cite{DBLP:journals/corr/HeZRS15}, Fast-RCNN \cite{DBLP:journals/corr/Girshick15}
και YOLO (You Only Look Once) \cite{DBLP:journals/corr/RedmonDGF15}, με τα
τελευταία δύο να επικεντρώνονται στο πρόβλημα της γρήγορης εκτέλεσης .
Συγκεκριμένα το δίκτυο YOLO έχει την ικανότητα να επεξεργαστεί εικόνες με
ταχύτητα 45 fps (frames per second) σε μονάδα GPU (NVIDIA Titan Χ).

